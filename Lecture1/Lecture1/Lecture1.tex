\documentclass[12pt, t]{beamer}
\input{~/Template/Academic/BeamerPPT/header.tex}

\author{Pingbang Hu}
\newcommand{\Title}{Honor Mathematics Lecture 1}
\newcommand{\SubTitle}{\textbf{Logic}\\``Without logic, mathematics will fall apart... "}
\newcommand{\InstitutionS}{University of Michigan}
\newcommand{\InstitutionL}{University of Michigan}

\begin{document}

\begin{frame}
    \titlepage
    \begin{center}
        \includegraphics[height=2cm]{Figures/logo/logo2.png}
    \end{center}
\end{frame}

\begin{frame}
    \frametitle{Pay attention to following...}
    For studying Honor Mathematics series well...
    \begin{itemize}
        \item \textbf{Forget EVERYTHING you have learned in school before.}
        \item Do think more about the question in ``()''. \\e.g. ``(How to prove?)''
        \item You are welcome to ask questions in an adequate manner.
        \item The class is designed to be interactive. Don't be so shy!
        \item Focus more on the idea of a proof. Don't just "recite" everything.
    \end{itemize}
\end{frame}

\begin{frame}
    \frametitle{Overview}
    \begin{enumerate}
        \item Statement
        \item Logical Operation
        \item Truth Table
        \item Relations between Statements
        \item Logical Quantifiers
        \item Sets
        \item Set Operations
        \item Ordered Pairs
        \item Russel Antinomy
        \item Exercises
    \end{enumerate}
\end{frame}

\subsection{Statement}
\begin{frame}
    \frametitle{Statement}
    Statement, also called as proposition, is anything we can regard as being either \emph{true} or \emph{false}.

    Some important things about statements:
    \begin{itemize}
        \item True statement
        \item False statement
        \item \emph{Vacuous Truth}
        \item Statement variable
        \item Statement structure \\
              e.g. quantifier, predicate, specific value
    \end{itemize}
\end{frame}

\subsection{Logical Operation }
\begin{frame}
    \frametitle{Logical Operation}
    Well, statements are important, but it will not be in this case without logical operation to combine different statements together.

    \begin{table}
        \centering
        \begin{tabular}{cccc}
            \toprule
            \multicolumn{4}{c}{Logical Operation} \\
            \midrule
            $\neg$   &  & Negation                \\
            $\wedge$ &  & Conjunction             \\
            $\vee$   &  & Disjunction             \\
            \bottomrule
        \end{tabular}%
    \end{table}

    For example, we see that we can have \(3>2\) is a true statement, and \(x^3>0\) is not a statement since we can not decide whether
    it is true or not. In the above example, the variable \(x\) is so-called \emph{statement variable}.

    If you want to go deeper, Let us see another example.

\end{frame}

\begin{frame}
    \frametitle{Logical Operation}
    Consider the statement
    \begin{center}
        "For any number \(x\), \(x^3>0\)"
    \end{center}

    \vspace{1em}
    The first part of the statement is a \emph{quantifier}("for any number \(x\)"), while the second part is called a \emph{statement frame}
    or \emph{predicate}("\(x^3>0\)").

    \vspace{1em}
    A statement frame becomes a statement(which can then be either true or false) when the variable takes on a specific value.

\end{frame}

\begin{frame}
    \frametitle{Logical Operation}
    Back to logical operation. It is clear that we can have something like
    \[
        A\lor B
    \]
    where \(A\) and \(B\) are two statements. We called these kinds of expression as \emph{compound statement}.

    \vspace{1em}
    A compound statement that is always true is called a \emph{tautology}. On the other hand, if the compound statement is always false is
    called a \emph{contradiction}.

\end{frame}

\subsection{Truth Table}
\begin{frame}
    \frametitle{Truth Table}
    How to use Truth Table?
    \begin{itemize}
        \item Understand what the problem is about.
        \item Set up a Truth Table.
        \item Always cover all possible cases.
    \end{itemize}
    Just to show you how powerful the concept of Truth Table is, here is an example.
\end{frame}

\subsection{Example of real engineering application}
\begin{frame}
    \frametitle{Application of Truth Table}
    Have you ever consider how a computer add two number?\\
    A: It is implemented by logic gates, and then circuits.\\
    Introduce some logic gates.
    \begin{figure}
        \centering
        \includegraphics[width=7cm]{Figures/Logic.png}
    \end{figure}
\end{frame}


\begin{frame}
    \frametitle{Adder}
    Then, we can consider following:\\
    \begin{center}
        \textbf{Given two numbers as input, how to add them together?}\\
    \end{center}

    As we all know, circuits work with binary number, e.g. 01011...\\
    So, the question becomes
    \begin{center}
        \textbf{How to add two one digit binary numbers?}\\
    \end{center}

    The idea is quite simple, that is let the circuit \emph{mimic} the action when we perform addition.
    In other words, we want this circuit to cover all possible cases when we do the basic addition, namely\\
    \begin{align*}
        00+00=00 \quad 00+01=01 \quad 01+00=01 \quad 01+01=10
    \end{align*}
\end{frame}

\begin{frame}
    \frametitle{Adder}
    We consider how each result's digit will react to the changing of input by constructing a Truth Table as follows:
    \begin{figure}
        \centering
        \includegraphics[width=5cm]{Figures/Adder_T.png}
    \end{figure}
    Then if you done your assignment 1.4, you should know you can construct a \emph{logic equation} from this Truth Table for \textbf{Sum} and \textbf{Carry}.\\
\end{frame}


\begin{frame}
    \frametitle{Logic gates for Adder}
    \begin{align*}
        Sum=A'B+AB' \quad Carry=AB
    \end{align*}
    Then we can get following circuit
    \begin{figure}
        \centering
        \includegraphics[width=5cm]{Figures/HalfAdder.png}
    \end{figure}
    Which is so-called \emph{Half Adder}.
\end{frame}

\begin{frame}
    \frametitle{More about Adder \dots}
    You can design a more complicated circuit by using same method, like:
    \begin{figure}
        \centering
        \includegraphics[width=9cm]{Figures/FullAdder.png}
    \end{figure}
    Which is so-called \emph{Full Adder}. You will learn these interesting concept in VE270.
\end{frame}

\begin{frame}
    \frametitle{Relations between Statements}
    \begin{itemize}
        \item Implication
              \begin{equation*}
                  A \Rightarrow B
              \end{equation*}
        \item Equivalence
              \begin{equation*}
                  A \Leftrightarrow B
              \end{equation*}
        \item Contraposition
              \begin{equation*}
                  (A\Rightarrow B) \Leftrightarrow (\neg A \Leftarrow \neg B)
              \end{equation*}
    \end{itemize}
    Proof of the contraposition (\textbf{de Morgan rules}):
    \begin{figure}
        \centering
        \includegraphics[width=10cm]{Figures/TruthTable.png}
    \end{figure}
    And this is the concept of \emph{tautology}.
\end{frame}

\begin{frame}
    \frametitle{Relations between Statements}
    Something very important related to tautology is called \emph{vacuously true}. Consider this.

    \begin{example}
        Let \(M\) be the set of real numbers \(x\) such that \(x = x+1\). Then the statement
        \[
            \underset{x\in M}{\forall }\ x>x
        \]
        is true.
    \end{example}
    Why?

    \vspace{1em}
    It's essentially similar to say that "All pink elephants can fly." is a true statement, because it is impossible to find
    a pink elephant that can't fly.
\end{frame}

\begin{frame}
    \frametitle{Exercise}
    Let $P,Q$ be two sets such that $P \subseteq Q$. Then what is the relation between these
    two statements?
    \\ \center Statement A: $x \in P$. Statement B: $x \in Q$.

\end{frame}

\begin{frame}
    \frametitle{Logical Quantifiers}
    \begin{table}
        \centering
        \resizebox{12cm}{!}{%
            \begin{tabular}{ccc}
                \toprule
                \multicolumn{3}{c}{Logical Quantifiers}                                                                \\
                \midrule
                Sign                           & Type               & Interpretation                                   \\
                \hline
                $\forall$                      & universal          & for any; for all                                 \\
                $\exists$                      & existential        & there exist; there is some                       \\
                $\forall \dots \forall \dots $ & nesting quantifier & for all \dots for all \dots                      \\
                $\exists \dots \exists \dots $ & nesting quantifier & there exists \dots (such that) there exist \dots \\
                .                              & .                  & .                                                \\
                .                              & .                  & .                                                \\
                .                              & .                  & .                                                \\
                \bottomrule
            \end{tabular}%
        }
    \end{table}

    \vspace{1em}
    We see that in order to use logical quantifiers properly, we need a \emph{domain} of our predicates. This is where set is needed.
\end{frame}

\begin{frame}
\frametitle{Sets}
\begin{enumerate}
\item What is a set?
\item Common set types
\begin{itemize}
\item Empty set: $\varnothing \coloneqq \{x\colon x\neq x\}$
\item Total set
\item Subset
\item Proper subset
\item Power set
\end {itemize}

\end{enumerate}

\par Simple question:
\center Why is $\varnothing$ a subset for any set $X$?
\end{frame}

\begin{frame}
    \frametitle{Example}
    let $A:= \{4,5,6\}$ be a set.
    \begin{itemize}
        \item The total set can be $\mathbb{N}$
        \item $B:= \{4,5,5,6,6\}=A$
        \item $C=\{1,5\}\subseteq A$
        \item $P:=\{\emptyset,\{4\},\{5\},\{6\},\{4,5\},\{5,6\},\{4,6\},\{4,5,6\}\}$
              \\\hspace{1em} is the power set of $A$. (What is the cardinality of $A$ ?)
    \end{itemize}
\end{frame}

\begin{frame}
    \frametitle{Naive Set Theorey: Sets via Predicates}
    We see that we want to be able to talk about the \emph{collection of objects}. However, it's hard to strictly define what an
    "object" or a "collection" is. The problem with what we are using, namely naive set theory is that any attempt to make a formal
    definition will lead to a contradiction.

    However, we are not going into the detail for this, we just stick with what we need, and naive set theory is enough. We indicate that an
    object, called an \emph{element} \(x\) is part of a collection, called a \emph{set} \(X\) by writing \(x\in X\).
    We characterize this relation by relating this with a predicate \(P(x)\) such that
    \[
        x\in X \iff P(x).
    \]
    We write such a set \(X\) in the form of
    \[
        X = \left\{ x\colon P(x) \right\}.
    \]
\end{frame}

\begin{frame}
    \frametitle{Set Operations}
    Define
    \begin{equation*}
        A:=\{1,2\} \quad B:=\{2,3\} \quad M:=\{1,2,3,4,5\}
    \end{equation*}
    \begin{table}
        \centering
        \resizebox{6cm}{!}{%
            \begin{tabular}{ccc}
                \toprule
                \multicolumn{3}{c}{Set Operations}          \\
                \midrule
                $A\cup B$      & Union        & $\{1,2,3\}$ \\
                $A\cap B$      & Intersection & $\{2\}$     \\
                $A\setminus B$ & Difference   & $\{1\}$     \\
                $A^c $         & Complement   & $\{3,4,5\}$ \\
                \bottomrule
            \end{tabular}%
        }
    \end{table}

    \par Simple question:
    \center What is $M^c$ ? Also, what is $\varnothing ^c$ ?
\end{frame}

\begin{frame}
    \frametitle{Ordered Pairs}
    \begin{itemize}
        \item What is an ordered pair?
        \item What is the difference between ordered pair and set?
        \item Concept of \emph{Cartesian product}.
    \end{itemize}

    Question: How can we show the order relation by what we have defined, namely only use set?
\end{frame}

\begin{frame}
    \frametitle{Russel Antinomy}
    There exist several paradoxes in native set theory, including:\\
    \begin{enumerate}
        \item Russel Antinomy
        \item Cantor's paradox
        \item Burali-Forti paradox
    \end{enumerate}
    \par \hspace{1em} The above paradoxes illustrate the fundamental flaw of our naive theory, namely
    it's not \textbf{well-defined}.\\
    \par \hspace{1em} However, these problems can be solved if we replaced naive set theory by a
    \emph{modern axiomatic set theory}, but the detail about it is beyond our scope. However, we'll show Russel Antinomy mathematically rather
    than tell that barber story.

    \vspace{1em}
    If you're interesting, you can see ZF-set theory, which can be considered the first success attempt for formalizing set theory.
\end{frame}

\begin{frame}
    \frametitle{Russel Antinomy}
    \begin{theorem}
        Russel Anatinomy. The predicate \(P(x)\colon x\notin x\) does not define a set
        \[
            A = \{x\colon P(x)\}.
        \]
    \end{theorem}

    \begin{proof}
        If \(A = \{x\colon x\notin x\}\) were a set, then we should be able to tell if for any set \(y\) such that it is in \(A\) or not.

        But we see that if we consider \(y = A\), then
        \begin{itemize}
            \item Assume \(A\in A\). Then \(P(A)\), which means \(A\notin A\). \contd
            \item Assume \(A\notin A\). Then \(\not P(A)\), which means \(A\in A\). \contd
        \end{itemize}
        We see that we can't decide \(A\in A\) or \(A\notin A\), hence \(A\) can not be a set.
    \end{proof}
\end{frame}

\begin{frame}
    \frametitle{Exercises}
    1. Let $A, B, C$ be three statements. Use truth table to prove that
    \begin{equation*}
        (A\vee B)\wedge C \equiv (A\wedge C)\vee (B\wedge C)
    \end{equation*}
\end{frame}

\begin{frame}
    \frametitle{Exercises}
    2. Let $A, B, C$ be three sets. Prove that
    \begin{equation*}
        (A\cup B)\cap C =(A\cap C)\cup (B\cap C)
    \end{equation*}
    From Exercise 1 \& 2, we can see that sets and statements are similar.
\end{frame}

\begin{frame}
    \frametitle{Exercises}
    3. Check whether the following sentences are true statement, false statement, or not a statement.
    \begin{itemize}
        \item $\forall x, y\in \mathbb{R}, x^2+y^3\geq 0$
        \item Let  $f(a)=a^4$, then $f(0)>0$
        \item For any $a\in \mathbb{R}, a^4>0$
        \item An African Elephant is very big.
        \item Let $A, B$ be two statements, then $(A\vee B)\Leftrightarrow\neg(\neg A\wedge\neg B)$
    \end{itemize}
    Simple question
    \begin{center}
        Rewrite above sentences in quantifiers form if they are statement.
    \end{center}

\end{frame}

\begin{frame}
    \frametitle{Exercises}
    4. Use quantifiers to rewrite the following definition of covergence:
    \par \vspace{2em} \hspace{1em} Let $(a_n)_{n\in \mathbb{N}}$ be a real sequence. If for some fixed $c \in \mathbb{R}$,
    for any $a>0$, there is an $N\in \mathbb{N}$, such that for all $n>N$, $|a_n-c|<a$,\\ then we
    say $(a_n)$ converges.
\end{frame}

\begin{frame}
    \frametitle{Reference}
    Reference.
    \begin{itemize}
        \item Exercises from 2019--Vv186 TA-Zhang Leyang.
        \item Figure for circuits from Ve270 T2-Logic Gate Slides.
    \end{itemize}
\end{frame}

\begin{frame}
    \frametitle{End}
    \vspace{2cm}
    \Huge \center  Have Fun and Learn Well!
\end{frame}


\end{document}